\documentclass{article}
	\usepackage{amsmath}
	\usepackage{amssymb}
\begin{document}
	The Newton's second law is F=ma.

	The Newton's second law is $F=ma$.

	The Newton's second law is $$F=ma$$
 
	The Newton's second law is \[F=ma\]

	Greek Letters $\eta$ and $\mu$

	Fraction $\frac{a}{b}$

	Power $a^b$

	Subscript $a_b$

	Derivate $\frac{\partial y}{\partial t}$

	Vector $\vec{n}$

	Bold $\mathbf{n}$
	
	To time differential $\dot{F}$

	Matrix (lcr here means left, center or right for each column)
	\[
		\left[
			\begin{array}{lcr}
				a1 & b22 &c333 \\
				d444 & e555555&f6
			\end{array}
		\right]
	\]

	Equations( here \& is the symbol for aligning different rows)
	\begin{align}
		a+b&=c\\
		d&=e+f+g
	\end{align}
	
	\[ 
		\left\{
			\begin{aligned}
				&a+b=c\\
				&d=e+f+g
			\end{aligned}
		\right.
	\]
	
	%数学公式
	%正文公式或者方程的生成:
	%1、\begin{math} 公式文本 \end{math}
	%2、\(公式文本\)
	%3、$公式文本$
	%显示公式或者方程的生成
	%\begin{displaymath} 公式文本 \end{displaymath} %无编号
	%\begin{equation} 公式文本 \end{equation} %自动加上公式编号
	%\[公式文本\] %同displaymath
	%创建多行公式
	%\begin{eqnarray} 公式文本 \end{eqnarray} %自动加上公式编号
	%\begin{eqnarray*} 公式文本 \end{eqnarray*} %无编号
	%数学公式的主要组成
	%显示公式中的{}:\{,\}
	$M(s)<M(t)<|M|=m$
	$y''=c\{f[y',y(x)] + g(x)\}$
	%指数和指标:^(把紧接下来的字符作为指数),_(把紧接下来的字符作为指标)
	\begin{displaymath}
		x^2 a_n x^n_i x_i^n
		x^{2n} x_{2y} a_{i,j,k}^{-n!2}
		x^{y^2} x^{y_1}
		A^{x_i^2}_{j^{2n}_{n,m}}
	\end{displaymath}
	%分数:/,\frac{分子}{分母}
	$(n+m)/2$
	$\frac{1}{x+y}$
	$\frac{a^2 - b^2}{a+b} = a-b$
	\[ \frac{\frac{a}{x-y} + \frac{b}{x+y}}{1+\frac{a-b}{a+b}} \]
	%方根
	%\sqrt[开方数]{参数} 开方数默认2
	$\sqrt[3]{8} = 2$
	$\sqrt{a}$
	%求和(\sum)与积分(\int)
	$\sum_{i=1}^n$
	$\int_a^b$
	$\int\limits_{x=0}^{x=1}$
	$\int y\,dx$
	$\int f(z)\,dz$
	%连续点,省略号
	%\ldots 偏下的点 \cdots 中间点 \vdots 竖直点 \ddots 对角点
	$a_0, a_1,\ldots,a_n$
	$a_0 + a_1 + \cdots + a_n$
	%\ldots命令可以用在普通的文本模式中,其余三条命令只能用在数学模式中,文本模式中,\dots命令可以代替\ldots
	%数学符号
	%希腊字母:只能用在数学模式中
	\begin{displaymath}
		\alpha \theta o \tau
		\beta \vartheta \pi \upsilon
		\gamma \iota \varpi \phi
		\delta \kappa \rho	\varphi
		\epsilon \lambda \varrho \chi
		\varepsilon \mu \sigma \psi
		\zeta \nu \varsigma \omega
		\eta \xi
	\end{displaymath}
	%大写字母
	\begin{displaymath}
		\Gamma \Lambda \Sigma \Psi
		\Delta \Xi \Upsilon \Omega
		\Theta \Pi	\Phi
	\end{displaymath}
	$\mathnormal{\Gamma\Pi\Phi}$ %斜体
	%花体字母
	$\mathcal{A, B, C,...,Z}$
	%二元运算符
	\begin{displaymath}
		\pm \cap \circ \bigcirc
		\mp \cup \bullet \Box
		\times \uplus \diamond \Diamond
		\div \sqcap \lhd \bigtriangleup
		\cdot \sqcup \rhd \bigtriangledown
		\ast \vee \unlhd \triangleleft
		\star \wedge \unrhd \triangleright
		\dagger \oplus \oslash \setminus
		\ddagger \ominus \odot \wr
		\amalg \otimes
	\end{displaymath}
	%关系运算符及其否定
	\begin{displaymath}
		\le \leq \ge \geq \neq \sim
		\ll \gg \doteq \simeq
		\subset \supset \approx \asymp
		\subseteq \supseteq \cong \smile
		\sqsubset \sqsupset \equiv \frown
		\sqsubseteq \sqsupseteq \propto \bowtie
		\in \ni \prec \succ
		\vdash \dashv \preceq \succeq
		\models \perp \parallel \mid |
	\end{displaymath}
	%斜杠贯穿符号可用命令\not
	\begin{displaymath}
		\not\in \notin
	\end{displaymath}
	%箭头与指针
	\begin{displaymath}
		\leftarrow \gets \longleftarrow \uparrow
		\Leftarrow \Longleftarrow \Uparrow
		\rightarrow \to \longrightarrow \downarrow
		\Rightarrow \Longrightarrow \Downarrow
		\leftrightarrow \longleftrightarrow \updownarrow
		\Leftrightarrow \Longleftrightarrow \Updownarrow
		\mapsto \longmapsto \nearrow
		\hookleftarrow \hookrightarrow \searrow
		\leftharpoonup \rightharpoonup \swarrow
		\leftharpoondown \rightharpoondown \nwarrow
		\rightleftharpoons \leadsto
	\end{displaymath}
	%其他各类符号
	\begin{displaymath}
		\aleph \prime \forall \Box
		\hbar \emptyset \exists \Diamond
		\imath \nabla \neq \triangleright
		\jmath \surd \flat \clubsuit
		\ell \partial \natural \diamondsuit
		\wp \top \sharp \heartsuit
		\Re \bot \| \spadesuit
		\Im \vdash \angle \Join
		\mho \dashv \backslash \infty
	\end{displaymath}
	%\limits 强迫上下限放在符号的上方和下方
	%\nolimits 上下限在位于符号的旁边
	\[\oint^\infty_0 \oint\limits^\infty_0 \] \\
	\[\prod^n_{\nu=0} \prod\nolimits^n_{\nu=0} \] \\
	%函数名:加上命令字符\
	\begin{displaymath}
		\arccos \cosh \det \inf \limsup \Pr \tan \\
		\arcsin \cot \dim \ker \ln \sec \tanh \\
		\arctan \coth \exp \lg \log \sin \\
		\arg \csc \gcd \lim \max \sinh \\
		\cos \deg \hom \liminf \min \sup \\
	\end{displaymath}
	$ a \bmod b$ \\
	$ y \pmod{a+b}$
	%数学重音
	\begin{displaymath}
		\hat{a} \breve{a} \grave{a} \bar{a} \\
		\check{a} \acute{a} \tilde{a} \vec{a} \\
		\dot{a} \ {a} \\
		\widehat{a} \widetilde{a} \\
	\end{displaymath}
	$\vec{\imath} + \tilde{\jmath}$ \\
	$\widehat{1-x}=\widehat{-y}$ \\
	$\widetilde{xyz}$ \\
	%括号符号的尺寸自动调整
	%\left 左括号 部分公式 \right 右括号
	\[ \left[ \int + \int \right]_{x=0}^{x=1} \] \\
	\[ y= \left\{\begin{array}{r@{\quad:\quad}l}-1 & x < 0 \\ 0 & x=0 \\ +1 & x > 0 \end{array} \right.\] \\
	%\left \right 命令必须成对出现,如果没有right,只能用'.'来表示看不见的括号符号
	%公式中的普通文本
	%\mbox{普通文本}
	\[X_n = X_k \qquad \mbox{ if and only if} \qquad Y_n = Y_k \quad \mbox{and} \quad Z_n = Z_k \]
	$\mathbf{B}^0(x)$ \quad $\mathsf{T}^i_j$
	$\mathnormal{differ} \ne \mathit{differ}$
	%矩阵和域(行列式、方程组等,统称为域
	%域用array环境生成
	\[\begin{array}{*{3}{c@{\:+\:}}c@{\;=\;}c} %\begin{array}{c@{\:+\:}c@{\:+\cdots+\:}c@{\;+\;}c}
		a_{11}x_1 & a_{12}x_2 & \cdots & a_{1n}x_n & b_1\\
		a_{21}x_1 & a_{22}x_2 & \cdots & a_{2n}x_n & b_2\\
			\multicolumn{5}{c}{\dotfill}			\\
		a_{n1}x_1 & a_{n2}x_2 & \cdots & a_{nn}x_n & b_n
	\end{array} \]
	%@{文本}在相邻的两列间插入文本的内容
	%\:和\;用来生成数学模式中小的水平间距
	%*{3}{c@{\:+\:}}列定义c@{\:+\:}的三次重复,c规定了列文本是居中排列的
	\[ \left(\begin{array}{c}
		\left|\begin{array}{cc}
		x_{11} & x_{12} \\ x_{21} & x_{22}
		\end{array} \right|
		x \\y \end{array} \right) \]
	\[ \sum_{p_1<p_2<cdots<p_{n-k}}^{(1,2,\ldots,n)}
		\Delta_{\begin{array}{l}
			p_1p_2\cdots p_{n-k} \\ p_1p_2\cdots p_{n-k}
		  \end{array}}
	\sum_{q_1<q_2<\cdots<q_k} \left| \begin{array}{llcl}
			a_{q_1q_1} & a_{q_1q_2} & \cdots & a_{q_1q_k} \\
			a_{q_2q_1} & a_{q_2q_2} & \cdots & a_{q_2q_k} \\
				\multicolumn{4}{c}\dotfill \\
			a_{q_kq_1} & a_{q_kq_2} & cdots & a_{q_kq_k}
				\end{array} \right| \]
	\[ x - \begin{array}{c}
		a_1 \\ \vdots \\ a_n \end{array}
	- \begin{array}[t]{cl}
		u - v & 10 \\
		u + v & \begin{array}[b]{r}
			12\\-120 \end{array}
		\end{array}		\]
	%公式上下的直线
	%\overline{部分公式} 和 \underline{部分公式}
	%\underline也可在普通文本模式中生成下划线,\overline只能用在数学模式中
	\[ \overline{\overline{a}^2 + \underline{xy} + \overline{\overline{z}}} \]
	%\overbrace{部分公式} 和 \underbrace{部分公式} 在部分公式的上方或下方放上一个水平的大括号
	\[ \overbrace{a + \underbrace{b + c} + d} \]
	\[ \underbrace{a + \overbrace{b + \cdots + y}^{123} + z}_{\alpha\beta\gamma} \]
	%堆积符号
	%\stackrel{上部符号}{下部符号}
	$\vec{x} \stackrel{\mathrm{def}}{=}(x_1,\cdots, x_n)$ \\
	$ A \stackrel{\alpha'}{\longrightarrow} B ... $\\
	%其他的数学命令
	%{顶部公式 \atop 底部公式} {顶部公式 \choose 底部公式}
	%\choose命令,该结构被包含在小括号内
	%\begin{array}{c}顶部行 \\ 底部行 \end{array} 同(atop)
	%\left(\begin{array}{c} 顶部行 \\ 底部行 \end{array}\right) 同(choose)
	\[ {n+1 \choose k} = {n \choose k} + {n \choose k-1} \]\\
	\[ \prod_{j\ge0}\left( \sum_{k\ge0} a_{jk}z^k \right) = 
		\sum_{n\ge0} z^n \left(\sum_{k_0, k_1, \ldots\ge0 \atop
		k_0 + k_1 +\cdots = 0} a_{0k_0} a_{1k_1}\ldots \right) \]\\
\end{document}
