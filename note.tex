
\documentclass[12pt,twoside,a4paper]{article}   %book(parts, sections, subsections, chapters...),report(parts, sections, subsections, chapters...),article(parts, sections, subsections...),letter
\NeedsTeXFormat{LaTeX2e}[1994/06/01]
%调用宏包,宏包围文件的基本名,宏包被保存在一个后缀为.sty的文件中
%\usepackage{宏包}
%\usepackage[选项1, 选项2, ...]{宏包1, 宏包2, ...}
%\pagestyle{样式}
\pagestyle{myheadings} \markright{Exercises} %plain,empty,headings,myheadings
\thispagestyle{plain} %当前页的样式
%headings和myheadings,出现在页眉中的信息用下面的声明确定
%\makeright{右边纸页眉}
%\makeboth{左边纸页眉}{右边纸页眉}  文档类的选项为\twoside
\pagenumbering{arabic} %roman Roman alpha Alpha
\setcounter{page}{2} %页码不从1开始
\hyphenation{man-u-script com-pu-ter gym-na-sium}
\begin{document}
	%标题页形式
	%1
	%\begin{titlepage}
	%How to Write DVI Drivers
	%\end{titlepage}
	%2
	%\title{How to Write DVI Drivers}
	%\author{name}
	%\date{2017.2.23}
	%\maketitle
	%
	\twocolumn
	\title{How to Write DVI Drivers}
	\author{Helmut kopka\thanks{tel. %\thanks{脚注文本},可以出现在title,author,date文本的任何地方
		05556-401-451 FRG}\\ %\\换行
		Max-Planck-institut\\
		f\"ur Amronomie
		\and %\and分开多个作者
		PhillipG,Hardy
		\thanks{tel.
		319-824-7134 USA}\\
		University\\of Iowa}
	\maketitle %以出现在\title,\author, \date,\thanks中的条目生成标题页
	\pagebreak[1]	%参数0-4
	\onecolumn
	%摘要
	\begin{abstract}
  	Abstract text
	\end{abstract} %book没有摘要
	\newpage
	\setcounter{tocdepth}{3} %出现在目录表中章节深度
	\tableofcontents %目录表的生成和显示
	%生成图和表的清单列表
	\listoffigures
	\nopagebreak[1]
	\listoftables
	\clearpage
	\today
	\section{Hello China} China is East Asia.
		\subsection{Hello Beijing} Beijing is the capital of china.
			\subsubsection{Hello Dongcheng District}
				\paragraph{Tian'anmen Square} is in the center of Beijing
					\subparagraph{Chairman Mao} is in the center of Tian'anmen Square
			\subsection{Hello Guangzhou}
				\paragraph{Sun Yat-sen University} is the best university in Guangzhou.
	\section{else}
	%\part,\chapter %article中不可用,\subsection,\paragraph,\section,\subsubsection,\subparagraph
	%book和report中,\chapter章标题是单个数字为编号,且总是开始新页,\section得到的是两个数字
	
	%断词\-
	man\-u\-script
	manu-\linebreak script

	%特殊字符:
	\#\$\&\_\"\%\^\{\}\\
	``father-in-law'' `--' ---\\
	%负号 
	$-$\\

	i.~e.,Prof.~Jones,Phys.\ Rev.\\ %接在小写字母后的句号认为是句子的结束,这时会插入额外的单词间距
	this is sentence ends with NASA\@.\\ %紧接在大写字母后的句号不任务是句子的介绍,而认为是缩写,加上\@才是结束
	\frenchspacing %不增加某些情形中的额外单词间距,可以不用\@
	\nonfrenchspacing  %恢复到原来的方式
	``\,`Beginning' and`'End'\,"
	%倾斜校正\/
	{\slshape slanted\/} spacing\\ %当倾斜字母后接逗号或句号时,可以不加
	%\textsl \textit会自动加上倾斜校正
 	\textsl{slanted}\\
 	%如果不希望这种校正,可以在倾斜字母后用命令\nocorr
	%取消连写\/
 	self\/ful			selfful\\
 	%插入任意距离
 	This is\hspace{1cm}1cm\\	
 	This is \hspace{1cm}1cm\\
 	This is \hspace{1cm} 1cm\\
 	\hspace*{3em} This is 1cm\\ %em为当前字样中字母M的宽度
	\vspace{1cm}
 	%没有*,可以使得间隔位于两行之间时,就去掉这个空档,如同在一行的开头,空格要去掉一样
 	%*形式,不管任何情况,都会插入空白
 	Left \hfill Right\\ %\hfill是\hspace{\fill}的缩写
 	Left \hfill Center \hfill Right\\
	\quad insert spacing\\
	\qquad insert spacing\\
 	Start \dotfill\ Finish\\ %插入....,和\hfill用法一样
 	Left \hrulefill\ Center \hrulefill\ Right\\ %插入————————,和\hfill用法一样
 	Departure \dotfill\dotfill\dotfill\ 8:30 \hfill\hfill Arrival \hrulefill\ 11:45\\
 	%断行\\[距离],\\*[距离],距离为额外间距 \newline
	%linebreak[数],\nolinebreak
	\vfill	%\bigskip   \medskip  \smallskip  \par
		%强调\em \emph
	This is the easiest way to {\em emphasize} short ...
	A more logical method of \emph{emphasizing} a word ...
	The {\em first}, second, and {\em third font switch}
	The {\em first, {\em second, and {\em third font switch}}}
	%字体尺寸的修改(从小到大)\tiny \scriptsize \footnotesize \small \normalsize \large \Large \LARGE \huge \Huge
	%\normalsize为标准尺寸10pt
	normal {\large large \Large larger} normal again
	normal large larger normal again
	\sl slanted {\Large larger}
	%自然行间距\baselineskip,每次修改字体尺寸,\baselineskip都会被重设为相应的自然行距,由\setlength所做的设置也就失效了
	\setlength{\baselineskip}{15pt}
	\renewcommand{\baselinestretch}{1.5} %对所有尺寸的字体都保持相应的间距
	%字体命令
	%family: \textrm{文本} \texttt{文本} \textsf{文本}
	%Shape: \textup{文本} \textit{文本} \textsl{文本} \textsc{文本}
	%Series: \textmd{文本} \textbf{文本}
	%默认值: \textnormal{文本}
	%强调:   \emph{文本}
	normal and \textbf{bold and \textsl{slanted} and back} again
	\textit {come italics\nocorr} without correction {\slshape italics \nocorr\textup{without} correction}
	%字符集与符号\symbol{数}
	\symbol{62} \symbol{28} \symbol{'34} \symbol{"1C}
	{\tt\symbol{'40}\symbol{'42}\symbol{'134}}
	\newpage
	%居中文本
	\begin{center}
	 first row ,second row, ...
	\end{center}
	\centering
	\centerline{text}
	%单边调整
	\begin{flushleft}
		Unilateral adjustment
	\end{flushleft}
	\begin{flushright}
		Unilateral adjustment
	\end{flushright}
	\raggedleft Unilateral adjustment\\
	\raggedright Unilateral adjustment\\
	%两边缩进
	\begin{quote}
		Indent both sides
	\end{quote}
	\begin{quotation}
		Indent both sides
	\end{quotation}
	%诗歌缩进
	\begin{verse}
		verse
	\end{verse}
	%列表
	\begin{itemize}	
		\item List text
	\end{itemize}
	\begin{enumerate}
		\item list text
	\end{enumerate}
	\begin{description}
		\item list text
	\end{description}
	%广义列表
	\begin{list}{标准标签}{列表声明}
		\item list item %列表中的项
	\end{list}
	\newcounter{fig}
	\begin{list}{\bfseries\upshape Figure \arabic{fig}:}
		{\usecounter{fig}
		\setlength{\labelwidth}{2cm}\setlength{\leftmargin}{2.6cm}
		\setlength{\labelsep}{0.5cm}\setlength{\rightmargin}{1cm}
		\setlength{\parsep}{0.5ex plus0.2ex minus0.1ex}
		\setlength{\itemsep}{0ex plus0.2ex} \slshape}
		\item Page format with head, body, and foot, showing the meaning of the various elements involved.
		\item Format of a general list showing its elements.
		\item A demonstration of some of the possibilities for drawing pictures with \LaTeX.
	\end{list}
	%itermize样例
	\begin{itemize}
	\item aaaa
	\item bbbb
	\item cccc
	\end{itemize}
	%enumerrate样例
	\begin{enumerate}
	\item aaaa
	\item bbbb
	\end{enumerate}
	%description样例
	\begin{description}
	\item[objective] aaaa
	\item[Example] bbbb
	\item[else] cccc
	\end{description}
	%改变标签样式\item[样式],如\item[+]
	%参考文献
	\begin{thebibliography}{+} %标签样式
		\bibitem[Label]{Keyword}references
	\end{thebibliography}
	\newpage
	For additional information about \LaTeX\ and \TeX\ see \cite{lamport} and \cite{knuth, knuth:a}.
	\begin{thebibliography}{99}
		\bibitem{lamport} Leslie Lamport. \textsl{\LaTeX\ -- A Document Preparation System}.Addison--Wesley Co.,Inc.,Reading,MA,1985
		\bibitem{knuth} Donald E. Knuth.\textsl{Computers and Typesetting Vol.\ A--E}.Addison--Wesley Co., Inc.,Reading,MA,1984-1986
		\bibitem[6a]{knuth:a} Vol A:\textsl{The \TeX book}, 1984
		\bibitem[6b]{knuth:b} Vol B:\textsl{\TeX: The Program.},1986
	\end{thebibliography}
	\newpage
	The \LaTeX{} logo

	The \LaTeX\ logo

	The {\LaTeX} logo

	\begin{quote}
		{\bfseries This text appears in bold face}\\
		\setlength{\parindent}{0.5cm} this text appears in bold face
		\setlength{\textwidth}{12.5cm}
			\S 
			\ddag 
			\P
			\copyright 
			\pounds			
	\end{quote}
	%Practice 2.1
	Today (\today) the rate of exchange between the British pound and American dollar is \pounds 1 = \$1.63,an increase of 1\% over yesterday.

	\setlength{\parskip}{1.5ex} %两个段落之间的距离
	\setlength{\parindent}{0em} %段落第一行的缩进量, \noindent \indent取消或强迫出现缩进
	\renewcommand{\baselinestretch}{2} %两条基线之间正常间距的数值
	\setlength{\textwidth}{12.5cm} %文本行宽
	\begin{appendix}
		appendix text
	\end{appendix}
\end{document}

