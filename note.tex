
\documentclass[12pt,twoside,a4paper]{article}   %book(parts, sections, subsections, chapters...),report(parts, sections, subsections, chapters...),article(parts, sections, subsections...),letter
\NeedsTeXFormat{LaTeX2e}[1994/06/01]
%调用宏包,宏包围文件的基本名,宏包被保存在一个后缀为.sty的文件中
%\usepackage{宏包}
%\usepackage[选项1, 选项2, ...]{宏包1, 宏包2, ...}
%\pagestyle{样式}
\pagestyle{myheadings} \markright{Exercises} %plain,empty,headings,myheadings
\thispagestyle{plain} %当前页的样式
%headings和myheadings,出现在页眉中的信息用下面的声明确定
%\makeright{右边纸页眉}
%\makeboth{左边纸页眉}{右边纸页眉}  文档类的选项为\twoside
\pagenumbering{arabic} %roman Roman alpha Alpha
\setcounter{page}{2} %页码不从1开始
\hyphenation{man-u-script com-pu-ter gym-na-sium}
\begin{document}
	%标题页形式
	%1
	%\begin{titlepage}
	%How to Write DVI Drivers
	%\end{titlepage}
	%2
	%\title{How to Write DVI Drivers}
	%\author{name}
	%\date{2017.2.23}
	%\maketitle
	%
	\twocolumn
	\title{How to Write DVI Drivers}
	\author{Helmut kopka\thanks{tel. %\thanks{脚注文本},可以出现在title,author,date文本的任何地方
		05556-401-451 FRG}\\ %\\换行
		Max-Planck-institut\\
		f\"ur Amronomie
		\and %\and分开多个作者
		PhillipG,Hardy
		\thanks{tel.
		319-824-7134 USA}\\
		University\\of Iowa}
	\maketitle %以出现在\title,\author, \date,\thanks中的条目生成标题页
	\pagebreak[1]	%参数0-4
	\onecolumn
	%摘要
	\begin{abstract}
  	Abstract text
	\end{abstract} %book没有摘要
	\newpage
	\setcounter{tocdepth}{3} %出现在目录表中章节深度
	\tableofcontents %目录表的生成和显示
	%生成图和表的清单列表
	\listoffigures
	\nopagebreak[1]
	\listoftables
	\clearpage
	\today
	\section{Hello China} China is East Asia.
		\subsection{Hello Beijing} Beijing is the capital of china.
			\subsubsection{Hello Dongcheng District}
				\paragraph{Tian'anmen Square} is in the center of Beijing
					\subparagraph{Chairman Mao} is in the center of Tian'anmen Square
			\subsection{Hello Guangzhou}
				\paragraph{Sun Yat-sen University} is the best university in Guangzhou.
	\section{else}
	%\part,\chapter %article中不可用,\subsection,\paragraph,\section,\subsubsection,\subparagraph
	%book和report中,\chapter章标题是单个数字为编号,且总是开始新页,\section得到的是两个数字
	
	%断词\-
	man\-u\-script
	manu-\linebreak script

	%特殊字符:
	\#\$\&\_\"\%\^\{\}\\
	``father-in-law'' `--' ---\\
	%负号 
	$-$\\

	i.~e.,Prof.~Jones,Phys.\ Rev.\\ %接在小写字母后的句号认为是句子的结束,这时会插入额外的单词间距
	this is sentence ends with NASA\@.\\ %紧接在大写字母后的句号不任务是句子的介绍,而认为是缩写,加上\@才是结束
	\frenchspacing %不增加某些情形中的额外单词间距,可以不用\@
	\nonfrenchspacing  %恢复到原来的方式
	``\,`Beginning' and`'End'\,"
	%倾斜校正\/
	{\slshape slanted\/} spacing\\ %当倾斜字母后接逗号或句号时,可以不加
	%\textsl \textit会自动加上倾斜校正
 	\textsl{slanted}\\
 	%如果不希望这种校正,可以在倾斜字母后用命令\nocorr
	%取消连写\/
 	self\/ful			selfful\\
 	%插入任意距离
 	This is\hspace{1cm}1cm\\	
 	This is \hspace{1cm}1cm\\
 	This is \hspace{1cm} 1cm\\
 	\hspace*{3em} This is 1cm\\ %em为当前字样中字母M的宽度
	\vspace{1cm}
 	%没有*,可以使得间隔位于两行之间时,就去掉这个空档,如同在一行的开头,空格要去掉一样
 	%*形式,不管任何情况,都会插入空白
 	Left \hfill Right\\ %\hfill是\hspace{\fill}的缩写
 	Left \hfill Center \hfill Right\\
	\quad insert spacing\\
	\qquad insert spacing\\
 	Start \dotfill\ Finish\\ %插入....,和\hfill用法一样
 	Left \hrulefill\ Center \hrulefill\ Right\\ %插入————————,和\hfill用法一样
 	Departure \dotfill\dotfill\dotfill\ 8:30 \hfill\hfill Arrival \hrulefill\ 11:45\\
 	%断行\\[距离],\\*[距离],距离为额外间距 \newline
	%linebreak[数],\nolinebreak
	\vfill	%\bigskip   \medskip  \smallskip  \par
		%强调\em \emph
	This is the easiest way to {\em emphasize} short ...
	A more logical method of \emph{emphasizing} a word ...
	The {\em first}, second, and {\em third font switch}
	The {\em first, {\em second, and {\em third font switch}}}
	%字体尺寸的修改(从小到大)\tiny \scriptsize \footnotesize \small \normalsize \large \Large \LARGE \huge \Huge
	%\normalsize为标准尺寸10pt
	normal {\large large \Large larger} normal again
	normal large larger normal again
	\sl slanted {\Large larger}
	%自然行间距\baselineskip,每次修改字体尺寸,\baselineskip都会被重设为相应的自然行距,由\setlength所做的设置也就失效了
	\setlength{\baselineskip}{15pt}
	\renewcommand{\baselinestretch}{1.5} %对所有尺寸的字体都保持相应的间距
	%字体命令
	%family: \textrm{文本} \texttt{文本} \textsf{文本}
	%Shape: \textup{文本} \textit{文本} \textsl{文本} \textsc{文本}
	%Series: \textmd{文本} \textbf{文本}
	%默认值: \textnormal{文本}
	%强调:   \emph{文本}
	normal and \textbf{bold and \textsl{slanted} and back} again
	\textit {come italics\nocorr} without correction {\slshape italics \nocorr\textup{without} correction}
	%字符集与符号\symbol{数}
	\symbol{62} \symbol{28} \symbol{'34} \symbol{"1C}
	{\tt\symbol{'40}\symbol{'42}\symbol{'134}}
	\newpage
	%居中文本
	\begin{center}
	 first row ,second row, ...
	\end{center}
	\centering
	\centerline{text}
	%单边调整
	\begin{flushleft}
		Unilateral adjustment
	\end{flushleft}
	\begin{flushright}
		Unilateral adjustment
	\end{flushright}
	\raggedleft Unilateral adjustment\\
	\raggedright Unilateral adjustment\\
	%两边缩进
	\begin{quote}
		Indent both sides
	\end{quote}
	\begin{quotation}
		Indent both sides
	\end{quotation}
	%诗歌缩进
	\begin{verse}
		verse
	\end{verse}
	%列表
	\begin{itemize}	
		\item List text
	\end{itemize}
	\begin{enumerate}
		\item list text
	\end{enumerate}
	\begin{description}
		\item list text
	\end{description}
	%广义列表
	\begin{list}{标准标签}{列表声明}
		\item list item %列表中的项
	\end{list}
	\newcounter{fig}
	\begin{list}{\bfseries\upshape Figure \arabic{fig}:}
		{\usecounter{fig}
		\setlength{\labelwidth}{2cm}\setlength{\leftmargin}{2.6cm}
		\setlength{\labelsep}{0.5cm}\setlength{\rightmargin}{1cm}
		\setlength{\parsep}{0.5ex plus0.2ex minus0.1ex}
		\setlength{\itemsep}{0ex plus0.2ex} \slshape}
		\item Page format with head, body, and foot, showing the meaning of the various elements involved.
		\item Format of a general list showing its elements.
		\item A demonstration of some of the possibilities for drawing pictures with \LaTeX.
	\end{list}

	\newenvironment{figlist}{\begin{list}
		{\bfseries\upshape Figure \arabic{fig}:}
		{\usecounter{fig}...{0ex plus0.2ex}\slshape}}
		{\end{list}}
	\begin{figlist}
		\item A
	\end{figlist}
	%平凡列表,相当于列表,标签是空的
	\begin{trivlist}
		\centering \item[] text(内部文本)
	\end{trivlist}
	%itermize样例
	\begin{itemize}
	\item aaaa
	\item bbbb
	\item cccc
	\end{itemize}
	%enumerrate样例
	\begin{enumerate}
	\item aaaa
	\item bbbb
	\end{enumerate}
	%description样例
	\begin{description}
	\item[objective] aaaa
	\item[Example] bbbb
	\item[else] cccc
	\end{description}
	%改变标签样式\item[样式],如\item[+]
	%定理型的声明 \newtheore{结构类型}{结构标题}[其他计数器] \newtheorem{结构类型}[编号来源]{结构名称}
	\newtheorem{theorem}{Theorem}
	\newtheorem{axiom}{Axiom}
	\begin{theorem}[Balzano--Weierstrass]
		Every ...
	\end{theorem}
	\begin{axiom}
		The natural numbers form ......
	\end{axiom} 
	%\newthoerem{subthrm}[theorem]{Sub-Theorem}
	%制表符 在任一行上都可以重设或增加制表位,如果有足够的\>命令跳到下一个制表位,用\=命令就会插入一个制表位,否则它会重设下一个制表位 \pushtabs保存制表位 \poptabs激活保存的制表位
	%tabbing环境中没有自动断行
	%tabbing环境中\hfill、\hrulefill、\dotfill命令没有作用
	\begin{tabbing}
		Type\qquad\= Quality\quad\=
		Color\quad\= Price\\[0.8ex]
		Paper \> med. \>white \> low \\
		Leather \> good \> brown \> high\\
		Card	\> bad \> gray \> med.\\
		Old column 1 \= Old Column 2\\
		Left Column \> Middle Col
		\= Extra col\\
		New col 1 \= New col 2 \>
		Old col 3\\
		Column 1 \> Column 2 \> Column 3\\
		%样本行 为了不显示样本行,该行文本不是用\\结束,而是用\kill
		\hspace*{3cm} \=sample column \= \hspace{4cm} \= \kill
		\a`o \a'o \a=o
	\end{tabbing}
	\`o \'o \=o
	\begin{tabbing}
		Graefruit; \= \kill
		Apples: \> consumed by: \=people\+\+\\
		horses \\
		and \' sheep\-\\
		reasonably juicy\-\\
		Grapefruits: \> a delicacy\\
		\pushtabs
		(see also: \= melons\\
		\> pumpkins)\\
		\poptabs
		Horses \> feed on \>apples
	\end{tabbing}
	\newpage
	%盒子:LR盒子、段落盒子、标尺盒子
	%LR盒子:水平的从左到右的有序材料组成
	%\mbox{文本} \makebox[宽度][位置]{文本}
	%\fbox{文本} \framebox[宽度][位置]{文本}
	%位置参数 l(文本左对齐) r(文本右对齐) s(伸展文本,已达到所定义的宽度)
	\makebox[3.5cm]{centered text}\\
	\framebox[3.5cm][r]{right justified}\\
	\framebox[3.5cm][s]{stretched\dotfill text}\\
	\framebox[2mm]{centered}\\
	\makebox[0pt][l]{/}S\\
	\framebox[6\totalheight]{Text}\\	%\width(盒子的自然宽度) \height(从基线到顶部的距离) \depth(从基线到底部的距离) \totalheight(\height加上\depth)
	\newsavebox{\boxname} %创建盒子
	\sbox{\boxname}{text}	%\savebox{\boxname}[宽度][位置]{文本}--保存盒子
	\usebox{\boxname}\\ %把保存的文本当作一个整体插入文档中
	\begin{lrbox}{\boxname}
		text
	\end{lrbox} %等价于\sbox{\boxname}{文本}
	%LR盒子的竖直移位
	%\raisebox{上移量}[高度][深度]{文本}
	Baseline \raisebox{1ex}{high} and \raisebox{-1ex}{low}
	and back again\\
	%子段盒子和小页
	%\parbox[位置]{宽度}{文本}
	%\begin{minipage}[位置]{宽度}文本 \end{minipage}
	%位置可取值:b(盒子的底边与当前基线对齐) t(顶行文本与当前基线对齐)
	\parbox{3.5cm}{\sloppy this is a 3.5 cm wide parbox.It is vertically ccentered on the}
	\hfill CURRENT LINE \hfill
	\parbox{5.5cm}{Narrow pages are hard to format. They usually produce many warning message on the terminal. The command
	{\tt\symbol{92}sloppy} can stop this.}\\
	\begin{minipage}[b]{4.3cm}
		The minipage environment creates a vertical box like the parbox command.The bottom line of this minipage is aligned with the
	\end{minipage}\hfill
	\parbox{3.0cm}{middle of this narrow parbox,which in turn is aligned with}
	\hfill
	\begin{minipage}[t]{3.8cm}
		the top line of the right hand minipage. It is recommended that the user experiment with the positioning arguments to get used to their effects.
	\end{minipage}
	%竖直摆放的问题
	\begin{minipage}[b]{8.5cm}
		\parbox[t]{4.3cm}{The boxes are made visible here by framing them,} \hfill
		\parbox[t]{3.0cm}{and by marking the baselines(the vertical alignment points) with black dots}\\
		\mbox{}\\	%恰好顶部对齐,或者第一行文本对齐
	\end{minipage}	
	of text\\
	%加入没有内容的第一行的方法:\mbox{} \\ , \mbox{}[-\baselineskip]
	%段落盒子:竖直堆积的行组成
	%具有指定高度的段落盒子
	%\parbox[位置][高度][内部位置]{宽度}{文本}
	%\begin{minipage}[位置][高度][内部位置]{宽度} 文本 \end{minipage}
	%高度参数可以同\makebox和\framebox中的宽度一样,使用\heght, \width, \depth, \totalheight
	%内部位置只有当给定了高度时才有意义,可取的值为t(要文本靠盒子的顶部), b(把文本推向盒子的底部), c(竖直居中), s(伸展文本以填满整个盒子)
	\begin{minipage}[t][2cm][t]{3cm}
		This is a minipage of height 2~cm with the text at the top.
	\end{minipage}\hrulefill
	\parbox[t][2cm][c]{3cm}{In this parbix, the text is centered on the same height.}\hrulefill
	\begin{minipage}[t][2cm][b]{3cm}
		In this third paragraph box, the text is at the bottom.
	\end{minipage}

	%标尺盒子:黑色填充的实心矩形,通常用来画水平或竖直线
	%\rule[提升]{宽度}{高度}
	\rule{8cm}{3mm}\\
	\fbox{\rule[-2mm]{0cm}{6mm}Text}
	%类似\parbox命令和minipage环境这样的竖直盒子也可以作为\sbox或者\savebox命令中的文本,从而保存起来,以后可以通过\usebox调用它们
	%盒子样式参数
	%对于有框盒子\fbox和\framebox,有两个用户可以重设的样式参数
	%\fboxrule 确定方框线的粗细
	%\fboxsep 设置方框与被包围文本之间的距离大小
	\newpage
	%表格
	%构造表格
	%\begin{array}[位置]{列} 行 \end{array}
	%\begin{tabular}[位置]{列} 行 \end{tabular}
	%\begin{tabular*}{宽度}[位置]{列} 行 \end{tabular*}
	%array环境只能用在数学模式中,参数意义如下
	%位置:竖直定位参数,取值:t,b
	%宽度:确定整体宽度。在这种情形下,[列]参数必须在第一项后面某个地方包含@表达式(@{extracolsep{\fill}},其它两种环境,整体宽度由文本内容确定
	%列:列格式参数,可能的列格式符号:l(列内容是左对齐的),r(列内容是右对齐的),c(列内容是居中的),p{宽}(该列的文本设置成具有给定宽的行),*{数}{列}(包含在列中的列格式被复制了数份,因此*{5}{|c}的结果与|c|c|c|c|c相同)
	%行:由表格的实际条目组成,每一水平行都由\\结束,这些行由一组彼此之间用&符号分开的列条目组成
	%\tabularnewline结束一行,明确的行结束符,而\\可以在一个行条目中结束一个文本行
	%为了使表格在页面上居中表格必须包围在\begin{center} 表格 \end{center}
	%表格样式参数
	%\tabclosep 插入在tabular和tabular*环境中两列间距离的一半
	%\arrayclosep 在array环境中相应于列间距的一半
	%\arrayrulewidth 表格中水平线与竖直线的粗细
	%\doublerulesep 双直线时两线之间的距离
	%可以用\setlength命令改变这些参数的值,如\setlength{\arrayrulewidth}{0.5mm}直线粗变为0.5mm
	%\arraystrecth 修改表格中的行间距,标准值为1,\renewcommand{arraystrecth}{因子}
	\begin{tabular}{rlcrrrcc}
	Position & Club & Games & W & T & L & Goals & Points\\[0.5ex]
	1 & Amesville Rockets & 33 & 19 & 13 & 1 & 66:31 & 51:15 \\
	2 & Borden Comets     & 33 & 18 &  9 & 6 & 65:37 & 45:21 \\
	... & ......          & .. & .. & .. & .. &...   & ...   \\
	17 & Quincy Giants    & 33 & 7  & 5  & 21 & 40::89 & 19:47 \\
	18 & Ralston Regulars & 33 & 3  & 11 & 19 & 37:74 & 17:49
	\end{tabular}
	\begin{tabular}{|r|l||c|rrr|c|c|}\hline
	Position & Club & Games & W & T & L & Goals & Points\\[0.5ex]
	\hline\hline
	1 & Amesville Rockets & 33 & 19 & 13 & 1 & 66:31 & 51:15 \\
	\hline
	2 & Borden Comets     & 33 & 18 &  9 & 6 & 65:37 & 45:21 \\
	\hline
	... & ......          & .. & .. & .. & .. &...   & ...   \\
	\hline
	17 & Quincy Giants    & 33 & 7  & 5  & 21 & 40::89 & 19:47 \\
	\hline
	18 & Ralston Regulars & 33 & 3  & 11 & 19 & 37:74 & 17:49 \\
	\hline
	\end{tabular}
	\begin{tabular}{|r|l||@{ 33 }|rrr|r@{:}l|r@{:}l|}\hline
	Position & Club & W & T & L & \multicolumn{2}{c}{Goals} & \multicolumn{2}{c}{Points}\\[0.5ex]
	4 & Daysdon Bombers     & 14 & 10 &  9 & 66 & 50 & 38 & 28 \\
	\hline
	\end{tabular}
	\begin{tabular}{|r|l||rrr|r@{:}l|r@{:}l|||c|}\hline
	\multicolumn{10}{|c|}{\bfseries 1st Regional Soccer League --- Final Results 1994/95} \\ \hline
	& \itshape Club & \itshape W & \itshape T & \itshape L & \multicolumn{2}{c}{\itshape Goals} & \multicolumn{2}{c}{\itshape Points} & \itshape Remarks \\ \hline\hline
	\end{tabular}
	\newpage
	%参考文献
	\begin{thebibliography}{+} %标签样式
		\bibitem[Label]{Keyword}references
	\end{thebibliography}
	\newpage
	For additional information about \LaTeX\ and \TeX\ see \cite{lamport} and \cite{knuth, knuth:a}.
	\begin{thebibliography}{99}
		\bibitem{lamport} Leslie Lamport. \textsl{\LaTeX\ -- A Document Preparation System}.Addison--Wesley Co.,Inc.,Reading,MA,1985
		\bibitem{knuth} Donald E. Knuth.\textsl{Computers and Typesetting Vol.\ A--E}.Addison--Wesley Co., Inc.,Reading,MA,1984-1986
		\bibitem[6a]{knuth:a} Vol A:\textsl{The \TeX book}, 1984
		\bibitem[6b]{knuth:b} Vol B:\textsl{\TeX: The Program.},1986
	\end{thebibliography}
	\newpage
	The \LaTeX{} logo

	The \LaTeX\ logo

	The {\LaTeX} logo

	\begin{quote}
		{\bfseries This text appears in bold face}\\
		\setlength{\parindent}{0.5cm} this text appears in bold face
		\setlength{\textwidth}{12.5cm}
			\S 
			\ddag 
			\P
			\copyright 
			\pounds			
	\end{quote}
	%Practice 2.1
	Today (\today) the rate of exchange between the British pound and American dollar is \pounds 1 = \$1.63,an increase of 1\% over yesterday.

	\setlength{\parskip}{1.5ex} %两个段落之间的距离
	\setlength{\parindent}{0em} %段落第一行的缩进量, \noindent \indent取消或强迫出现缩进
	\renewcommand{\baselinestretch}{2} %两条基线之间正常间距的数值
	\setlength{\textwidth}{12.5cm} %文本行宽
	\begin{appendix}
		appendix text
	\end{appendix}
\end{document}

