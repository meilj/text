
\documentclass[12pt,twoside,twocolumn,a4paper]{article}   %book(parts, sections, subsections, chapters...),report(parts, sections, subsections, chapters...),article(parts, sections, subsections...),letter
\NeedsTeXFormat{LaTeX2e}[1994/06/01]
%调用宏包,宏包围文件的基本名,宏包被保存在一个后缀为.sty的文件中
%\usepackage{宏包}
%\usepackage[选项1, 选项2, ...]{宏包1, 宏包2, ...}
%\pagestyle{样式}
\pagestyle{myheadings} \markright{Exercises} %plain,empty,headings,myheadings
\thispagestyle{plain} %当前页的样式
%headings和myheadings,出现在页眉中的信息用下面的声明确定
%\makeright{右边纸页眉}
%\makeboth{左边纸页眉}{右边纸页眉}  文档类的选项为\twoside
\pagenumbering{arabic} %roman Roman alpha Alpha
\setcounter{page}{2} %页码不从1开始
\begin{document}
	%标题页形式
	%1
	%\begin{titlepage}
	%How to Write DVI Drivers
	%\end{titlepage}
	%2
	%\title{How to Write DVI Drivers}
	%\author{name}
	%\date{2017.2.23}
	%\maketitle
	%
	\title{How to Write DVI Drivers}
	\author{Helmut kopka\thanks{tel. %\thanks{脚注文本},可以出现在title,author,date文本的任何地方
		05556-401-451 FRG}\\ %\\换行
		Max-Planck-institut\\
		f\"ur Amronomie
		\and %\and分开多个作者
		PhillipG,Hardy
		\thanks{tel.
		319-824-7134 USA}\\
		University\\of Iowa}
	\maketitle %以出现在\title,\author, \date,\thanks中的条目生成标题页
	%摘要
	\begin{abstract}
  	Abstract text
	\end{abstract} %book没有摘要
	\setcounter{tocdepth}{3} %出现在目录表中章节深度
	\tableofcontents %目录表的生成和显示
	%生成图和表的清单列表
	\listoffigures
	\listoftables
	\section{Hello China} China is East Asia.
		\subsection{Hello Beijing} Beijing is the capital of china.
			\subsubsection{Hello Dongcheng District}
				\paragraph{Tian'anmen Square} is in the center of Beijing
					\subparagraph{Chairman Mao} is in the center of Tian'anmen Square
			\subsection{Hello Guangzhou}
				\paragraph{Sun Yat-sen University} is the best university in Guangzhou.
	%\part,\chapter %article中不可用,\subsection,\paragraph,\section,\subsubsection,\subparagraph
	%book和report中,\chapter章标题是单个数字为编号,且总是开始新页,\section得到的是两个数字

	%特殊字符:
	\#\$\&\_\"\%\^\{\}

	``father-in-law'' `--' ---

	%负号 
	$-$

	selfful
	i.~e.,Prof.~Jones,Phys.\ Rev. %接在小写字母后的句号认为是句子的结束,这时会插入额外的单词间距
	this is sentence ends with NASA\@. %紧接在大写字母后的句号不任务是句子的介绍,而认为是缩写,加上\@才是结束
	\frenchspacing %不增加某些情形中的额外单词间距,可以不用\@
	\nonfrenchspacing  %恢复到原来的方式
	``\,`Beginning' and`'End'\,"
	
	\today

	The \LaTeX{} logo

	The \LaTeX\ logo

	The {\LaTeX} logo

	\begin{quote}
		{\bfseries This text appears in bold face}

		\setlength{\parindent}{0.5cm} this text appears in bold face
		\setlength{\textwidth}{12.5cm}
			\S 
			\ddag 
			\P
			\copyright 
			\pounds			
	\end{quote}
	%Practice 2.1
	Today (\today) the rate of exchange between the British pound and American dollar is \pounds 1 = \$1.63,an increase of 1\% over yesterday.

	\setlength{\parskip}{1.5ex} %两个段落之间的距离
	\setlength{\parindent}{0em} %段落第一行的缩进量
	\renewcommand{\baselinestretch}{2} %两条基线之间正常间距的数值
	\setlength{\textwidth}{12.5cm} %文本行宽
	\begin{appendix}
		appendix text
	\end{appendix}
\end{document}

