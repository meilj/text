%图形
\documentclass{article}
\begin{document}
	\title{Graphical}
	\maketitle
	%设置单位长度,默认值1pt
		%\setlength{\unitlength}{长度}
	%坐标原点在左下角
	%图形环境
		%\begin{picture}(x尺寸, y尺寸) 画图命令 \end{picture}
	%定位命令
		%\put(x-坐标, y-坐标){图形元素}
		%\multiput(x-坐标, y-坐标)(x-增量, y-增量){数}{图形元素}
		%(x-坐标, y-坐标)为安置坐标, \multiput把同样的图形元素生成数次,每次移动(x-增量, y-增量)
		%对于坐标项,小数点必须是句号,而不能是逗号
	%基本画图命令
	\section{rectangle}
		%图形中的盒子-矩形
			%\makebox(x-尺寸, y-尺寸)[位置]{文本}   %没有矩形框
			%\framebx(x-尺寸, y-尺寸)[pos]{text}	   %有矩形框
			%\dashbox{虚线尺寸}(x-尺寸, y-尺寸)[位置]{文本}  %矩形框为虚线
				%(x-尺寸, y-尺寸)为矩形的宽度和高度
				%位置定义了文本在盒子中的位置,取值如下:
					%[t] top-输入文本水平居中的位于盒子顶边的下面
					%[b] bottom-输入文本水平居中的位于盒子低边的上面
					%[l] left-输入文本竖直居中的位于盒子的左边
					%[r] right-输入文本竖直居中的位于盒子的右边
					%[s] stretch-输入文本竖直居中,但要水平伸展以充满整个盒子
					%可以两个组合使用:[tl] [tr] [bl] [br],顺序无关紧要
		\setlength{\unitlength}{1cm}
		\begin{picture}(4, 4)
			\put(0.0, 1.95){\framebox(2, 1.0)[t]{top center}}
			\put(3.0, 1.95){\framebox(2, 0.8)[lb]{bot, left}}
			\put(3.0, 3.2){\framebox(2, 0.6)[r]{center right}}
			\put(2.0, 0.3){\framebox(2, 0.6)[s]{center\hfill stretch}}
		\end{picture}
		\newline
		\begin{picture}(4, 5)
			\put(0.0, 1.95){\makebox(2, 1.0)[t]{top center}}
			\put(3.0, 1.95){\makebox(2, 0.8)[lb]{bot, left}}
			\put(3.0, 3.2){\makebox(2, 0.6)[r]{center right}}
			\put(2.0, 0.3){\dashbox{0.2}(2, 0.6)[s]{center\hfill stretch}}
		\end{picture}
		\newline
	%直线
	\section{line}
		%\line(x, y){长度} 
		%绘制直线的倾角由斜率(x, y)给定,x,y的取值规则:
		%1、竖直必须是整数(负数或整数均可) 2、只可以取值0,1,...,6。  3、在数对中的两数不能有公因子 
		\begin{picture}(6, 5)
			\thicklines
			\put(0, 0){\line(1, 0){6}}
			\put(0, 0){\line(0, 1){1}}
			\put(6, 0){\line(0, 1){0.5}}
			\put(1.0, 2.75){\line(2, -1){3.5}}
		\end{picture}
		\newline
	%箭头
	\section{Arrow}
		%\vector(x, y){长度} %参数与\line命令完全一样,而且取值只能是0,1,2,3,4
		\begin{picture}(5,2)\thicklines
			\put(5, 0){\vector(-1, 0){5}}
			\put(0, 0){\vector(1, 1){2}}
			\put(2, 2){\vector(3, -2){3}}
		\end{picture}
		\newline
	%圆
	\section{Circle}
		%\circle{直径} %轮廓线
		%\circle*{直径} %实心圆
		%相应的\put命令中的安置位置对应于圆心
		\begin{picture}(3, 1.6)
			\put(1, 1){\circle*{0.2}}
			\put(1, 1){\circle{1.2}}
			\put(1, 1){\vector(0,1){0.6}}
		\end{picture}
		\newline
	%卵形线与圆角(椭圆)
	\section{oval}
		%\oval(x-尺寸, y-尺寸)[部分]
		%相应的\put命令中的安置坐标对应于卵形线的中心
		%可省参数[部分]取值t,b,l,r,生成一半的卵形线,也可以是组合,生成四分之一的卵形线
		\begin{picture}(10, 5)
			%\put(3.0, 0.75){\oval(4.0, 1.5)}
			\put(1.75, 4.2){\oval(3.5, 1.2)[b]}
			\put(1.75, 2.6){\oval(3.5, 1.2)[t]}
			\put(1.75, 1.0){\oval(3.5, 1.2)[l]}
			\put(3.25, 1.0){\oval(3.5, 1.2)[r]}
			\put(7.0, 2.5){\oval(3.0, 1.0)[tl]}
			\put(7.5, 2.5){\oval(3.0, 1.0)[tr]}
			\put(6.0, 1.5){\oval(1.0, 2.0)[bl]}
			\put(8.5, 1.5){\oval(1.0, 2.0)[br]}
		\end{picture}
		\newline
	%竖直堆积文本
	\section{shortstack}
		%\shortstack[位置]{列}
		%[位置]参数取值l,r,c,标准值是c,表示居中,[列]表示输入文本,行与行之间用\\分开
		\begin{picture}(10, 5)
			\put(1.0, 0.5){\shortstack[l]{This\\spacing\\leaves\\some\\...}}
			\put(3.0, 0.5){\shortstack{Not\\really\\the\\best\\...}}
			\put(5.0, 0.5){\shortstack[r]{Single\\characters\\...}}
			\put(8.0, 0.8){\shortstack{L\\e\\t\\t\\e\\r\\s}}
			\put(9.0, 1.2){\shortstack{b\\e\\s\\t}}
		\end{picture}
		\newline
	%有框文本
	\section{framebox}
		%\framebox 生成有框盒子
		%\fbox 围绕文本画一个方框
		%\frame{图形元素} 生成有框的图形
		\begin{picture}(5,2)
			\setlength{\fboxsep}{0.25cm}
			\put(0, 0){\framebox(5, 2){}}
			\put(1, 1){\fbox{fitted frame}}
		\end{picture}
		\begin{picture}(10,2)
			\put(2.0, 0.5){\frame{TEXT}}
			\put(3.5, 0.0){\frame{\shortstack{W\\O\\R\\D}}}
			\put(4.0, 0.0){\frame{\vector(1, 1){1.0}}}
			\put(6.0, 0.0){\frame{\circle{1.0}}}
			\put(8.0, 0.0){\frame{\makebox[1cm][c]{\circle{1.0}}}}
			\put(9.0, 1.0){\framebox(1,1)[b]{\circle{1.0}}}
		\end{picture}
		\newline
	%曲线
	\section{curve}
		%\bezier{数}(x1, y1)(x2, y2)(x3, y3)
		%\qbezier[数](x1, y1)(x2, y2)(x3, y3)
		%画一条从点(x1, y1)到(x3, y3)的二次Bezier曲线, 而(x2,y2)是Bezier控制点,曲线实际上是用[数]+1个点画出来的
		\begin{picture}(4, 2)
			\qbezier(0, 0)(2, 2)(4, 1)
		\end{picture}
		\newline
	%其它图形命令与示例
	\section{else command}
		\subsection{linear thickness}
			%\thicklines  \thinlines
			%\linethickness{粗细} %可用于,直线、斜线、\framebox、\dashbox
		\subsection{nested graph}
			%\put(x-坐标, y-坐标){\setlength{\unitlength}{单位长度}\begin{picture}(x-尺寸, y-尺寸)...子图形..\end{picture}
			\begin{picture}(10.0, 6.6)
				\thicklines\put(0, 0){\framebox(10.0, 6.6){}}
				\put(5.0, 6.3){\makebox(0, 0){\bfseries The Outer Picture}}
				\thinlines
				\put(.5, .5){\setlength{\unitlength}{1mm}
					\begin{picture}(50, 25)
						\put(0, 0){\framebox(40, 25){Sub Picture 1}}
						\put(10, 20){\circle*{1}}
						\put(4, 4){\vector(-1, -1){4}}
						\put(5, 5){\makebox(0, 0)[lb]{(0,0) Origin 1}}
					\end{picture}}
				\put(5.5, 0.5){ ... Sub Picture 2 ...}
				\put(0.5, 3.5){ ... Sub Picture 3 ...}
				\put(5.5, 3.5){ ... Sub Picture 4 ...}
				\put(5, 0.1){\makebox(0, 0)[b]{\bfseries Bottom of Outer Picture}}
			\end{picture}

\end{document}
