%图形
\documentclass{article}
\begin{document}
	%设置单位长度,默认值1pt
		%\setlength{\unitlength}{长度}
	%坐标原点在左下角
	%图形环境
		%\begin{picture}(x尺寸, y尺寸) 画图命令 \end{picture}
	%定位命令
		%\put(x-坐标, y-坐标){图形元素}
		%\multiput(x-坐标, y-坐标)(x-增量, y-增量){数}{图形元素}
		%(x-坐标, y-坐标)为安置坐标, \multiput把同样的图形元素生成数次,每次移动(x-增量, y-增量)
		%对于坐标项,小数点必须是句号,而不能是逗号
	%基本画图命令
		%图形中的盒子-矩形
			%\makebox(x-尺寸, y-尺寸)[位置]{文本}   %没有矩形框
			%\framebx(x-尺寸, y-尺寸)[pos]{text}	   %有矩形框
			%\dashbox{虚线尺寸}(x-尺寸, y-尺寸)[位置]{文本}  %矩形框为虚线
				%(x-尺寸, y-尺寸)为矩形的宽度和高度
				%位置定义了文本在盒子中的位置,取值如下:
					%[t] top-输入文本水平居中的位于盒子顶边的下面
					%[b] bottom-输入文本水平居中的位于盒子低边的上面
					%[l] left-输入文本竖直居中的位于盒子的左边
					%[r] right-输入文本竖直居中的位于盒子的右边
					%[s] stretch-输入文本竖直居中,但要水平伸展以充满整个盒子
					%可以两个组合使用:[tl] [tr] [bl] [br],顺序无关紧要
	\setlength{\unitlength}{1cm}
	\begin{picture}(4, 4)
		\put(0.0, 1.95){\framebox(2, 1.0)[t]{top center}}
		\put(3.0, 1.95){\framebox(2, 0.8)[lb]{bot, left}}
		\put(3.0, 3.2){\framebox(2, 0.6)[r]{center right}}
		\put(2.0, 0.3){\framebox(2, 0.6)[s]{center\hfill stretch}}
	\end{picture}
	\newline
	\begin{picture}(4, 5)
		\put(0.0, 1.95){\makebox(2, 1.0)[t]{top center}}
		\put(3.0, 1.95){\makebox(2, 0.8)[lb]{bot, left}}
		\put(3.0, 3.2){\makebox(2, 0.6)[r]{center right}}
		\put(2.0, 0.3){\dashbox{0.2}(2, 0.6)[s]{center\hfill stretch}}
	\end{picture}
\end{document}
